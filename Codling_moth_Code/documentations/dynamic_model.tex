\section{Dynamic Model}

The Dynamic Model excel file is provided 
\href{http://ucanr.edu/sites/fruittree/How-to_Guides/Dynamic_Model_-_Chill_Accumulation/}{here} and its use is explained 
\href{http://ucanr.edu/sites/fruittree/files/49320.pdf}{here}.


\begin{enumerate}
\item \func{Fahrenheit\_to\_Celsius(temp\_cel)} converts Fahrenheit temp. to Celsius. Celsius is the proper form of temp. used in the model. 

\code{Input}:
\begin{itemize}
\item \vari{temp\_cel}  A real valued temp. or a column of a 
data frame containing the temp in the Celsius format.
\end{itemize}


\code{output}
\begin{itemize}
\item A real valued temp. in Fahrenheit or a column of
data frame.
\end{itemize}

%%%%%%%%%%%%%%%

\item \func{initiate\_data\_frame(col\_names, init\_temp\_c, const)} Creates a new data frame 
of size $2 \times 13$ and fills in the cells which are to be used
by dynamic model. It corresponds to rows 11 and 12 of the 
model given in the excel file.

\code{Input}:

\begin{itemize}
\item \vari{col\_names}  The names of columns of the data frame to be used in the model.

{\footnotesize {$ \vari{col\_names} = [ \text{date}, \:
                                    \text{ time}, 
                                     \text{temp\_c}, 
                                     \text{temp\_k}, 
                                     \text{ftmprt}, 
                                     \text{sr}, 
                                     \text{xi}, 
                                     \text{xs}, 
                                     \text{ak1}, 
                                     \text{Inter-S}, 
                                     \text{Inter-E},
                                     \text{delt}, 
                                     \text{Portions}] $}}

\item \vari{init\_temp\_c}. Initial temps. corresponding to cells C11 and C12 
of the excel file. $\vari{init\_{temp\_c}} = (15, 12)$.

\item \vari{const} An object of the class \vari{constants} containing 
constants of the model. They are given below and in 
the D1 through D8 cells of the excel
file.
%%%%%%%%%
\iffalse
\begin{align*}
\vari{const} 
&=  \begin{bmatrix}
           \vari{e0} \\
           \vari{e1} \\
           \vari{a0}\\
           \vari{a1}\\
           \vari{slp}\\
           \vari{tetmlt}\\
           \vari{aa}\\
           \vari{ee}
         \end{bmatrix}  =
      \begin{bmatrix}
           4.15E+03 \\
           1.29E+04\\
           1.40E+05 \\
           2.57E+18\\
           1.6\\
           277\\
           \vari{a0} / \vari{a1}\\
           \vari{e1} - \vari{e0}
         \end{bmatrix}
\end{align*}
\fi
%%%%%%%%%

\begin{table}[!htb]
\caption{\vari{const} object}
\begin{center}
    \begin{tabular}{| l | l | l | l | l | l| l| l| l | l | l | p{1cm} |}
     \hline
    \scriptsize{\texttt{e0}} & \scriptsize{\texttt{e1}} & \scriptsize{\texttt{a0}} & \scriptsize{\texttt{a1}} & \scriptsize{\texttt{slp}} & \scriptsize{\texttt{tetmlt}} & \scriptsize{\texttt{aa}} & \scriptsize{\texttt{ee}} \\ \hline
             \scriptsize{\texttt{4.15E+03}} & \scriptsize{\texttt{1.29E+04}} & \scriptsize{\texttt{1.40E+05}} & \scriptsize{v{2.57E+18}} & \scriptsize{\texttt{1.6}} & \scriptsize{\texttt{277}} & \scriptsize{\texttt{a0 / a1}} & \scriptsize{\texttt{e1 - e0}}  \\ \hline
    \end{tabular}
\end{center}
 \label{table:None}
\end{table}

\end{itemize}

\code{output:} A data frame of the following form. The numbers in the {\vari{sr}} column in the table \ref{table:None1} are long numbers I did not put them in there.

\begin{table}[!htb]
\caption{initial data frame to construct the model with.}
\vspace{-.1in}
\begin{center}
    \begin{tabular}{| l | l | l| l | l | l | l | l | l| l| l| l | p{1cm} |}
    \hline
    \scriptsize{date}  & \scriptsize{time} & \scriptsize{temp\_c} & \scriptsize{temp\_k}  &  \scriptsize{ftmprt} & \scriptsize{sr} & \scriptsize{xi} & \scriptsize{xs} & \scriptsize{ak1} & \scriptsize{Inter-S} & \scriptsize{Inter-E} & \scriptsize{delt} & \scriptsize{Portions} \\ \hline
     \scriptsize{\texttt{None}} & \scriptsize{\texttt{None}} & \scriptsize{\texttt{15} }& \scriptsize{\texttt{288}} & \scriptsize{\texttt{16.93}} & \scriptsize{\texttt{--}} & \scriptsize{\texttt{1}} & \scriptsize{v{.81}} & \scriptsize{\texttt{.09}} & \scriptsize{\texttt{0.00}} & \scriptsize{\texttt{.07}} & \scriptsize{\texttt{0.00}} & \scriptsize{0}\\ \hline
     \scriptsize{\texttt{None}}  & \scriptsize{\texttt{None}} & \scriptsize{\texttt{12}} & \scriptsize{\texttt{285}} & \scriptsize{\texttt{12.44}} & \scriptsize{\texttt{-}} & \scriptsize{\texttt{1}} & \scriptsize{\texttt{1.11}} & \scriptsize{\texttt{.06}} & \scriptsize{\texttt{.07}} & \scriptsize{\texttt{.13}} & \scriptsize{\texttt{0.00}} & \scriptsize{{0}}  \\ \hline
    \end{tabular}
\end{center}
 \label{table:None1}
\end{table}


\item \func{fill\_in\_the\_table(given\_table, const)}
 This function takes the \vari{const} object and \vari{given\_table} 
 as input and runs the model to fill in the 
 proper information that we need to compute the Chill Portions 
 which is the ultimate goal of the model. 
 
 \code{input:}
 \begin{itemize}
 \item \vari{given\_table} Is the data frame that contains 
 the first two rows, like the one given by 
Table \ref{table:None1} and the first three 
 columns, from row 3 to the end, are provided by 
 datalogger and are read off the disk. Anything from the column
 \code{temp\_k}  onward is computed and filled by this function.

\item \vari{const}: The object containing constants of the model
mentioned before.
 \end{itemize}
 
\code{output:} A complete table that has the Chill Portions for the 
data of out orchard.

\item \func{dynamic\_model(path\_to\_data, col\_names, init\_temp\_c, const)}

This function takes the path of the file we wish 
to compute the Chilling portions for, along with 
other inputs that we have already mentioned 
before, and produces the Chilling Portions.

\code{input:}
\begin{itemize}
\item \vari{path\_to\_data}: the path to the data location on the disk.
\item \vari{col\_names}: Name of the columns 
of the data frame, like mentioned before. These 
names has to be exact, because they are used for
computations in the model.\\

{\color{red}{\textsc{NOTE}}}: These data should have temp. in 
Celsius. And it is assumed the first three columns 
are \code{date}, \code{time} and \code{temp} respectively.\\

\item \vari{init\_temp\_c} Initial temp. as mentioned before.

\item \vari{const}: An object containing the constants of the model.
\end{itemize}

\code{output:} A data frame containing all information we need. 
(Shall I change this so that it just gives the \vari{Portions}?)

\end{enumerate}

