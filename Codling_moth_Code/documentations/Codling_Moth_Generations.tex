
\section{Codling Moth Generations}

All these functions are written by Giridhar.
% \item \func{
% \code{
% \vari{
% \texttt{
% \scriptsize{

\begin{enumerate}

\item \func{readbinarydata\_addmdy(filename, Nrecords, Nofvariables, ymd, ind)}
This function reads the binary data off the disk, 
generates a matrix containing [\code{precip, Max temp., Min temp., Wind speed}] 
adds the \code{ymd} to it and returns it.

\code{input:}
\begin{itemize}
\item \vari{filename} This is the full path to the file which includes the file name.
\item \vari{Nrecords}
\item \vari{Nofvariables}
\item \vari{ymd} Year, Month, Day values
\item \vari{ind}
\end{itemize}

\code{output:} [\code{precip, Max temp., Min temp., Wind speed, ymd}]

\hrule

Why not \code{readRDF}? why connection thing?

I do not know what the rest are yet. and I do not 
know why \code{Nrecords} and {\code{Nofvariables}}
are not found inside the function by looking at dimension of 
the file read off the disk! Are not they related?

\hrule


\pagebreak

%%%%%%%%%%%%%%%
\item \func{create\_ymdvalues(nYears, Years, leap.year)}

\code{input:}
\begin{itemize}
\item \vari{nYears}: Number of years. (Is not this just \code{length(\vari{Years})}?)
\item \vari{Years}
\item \vari{leap.year}: A binary vector indicating whether a year is a leap year or not.
\end{itemize}
\code{output:} \vari{ymd}, i.e. \code{Year, Month, Day}.
%%%%%%%%%%%%%%%

%%%%%%%%%%%%%%%
\item \func{add\_dd\_cumudd(metdata\_data.table, lower, upper)}

Computes the degree days (\code{dd}), cumulative degree days (\code{Cum\_dd}) and concatenates it to the \code{metdata\_data.table}

\code{input:}
\begin{itemize}
\item \vari{metdata\_data.table}: the data
\item \vari{lower}: lower threshold
\item \vari{upper}: upper threshold
\end{itemize}

\code{output:} \vari{[metdata\_data.table, dd, Cum\_dd]}
%%%%%%%%%%%%%%%

%%%%%%%%%%%%%%%
\item \func{CodlingMothRelPopulation(CodMothParams, metdata\_data.table)}

\code{input:}
\begin{itemize}
\item \vari{CodMothParams}

\item \vari{metdata\_data.table}
\end{itemize}

\code{output:}
It takes a data table, and parameters of the model such as cumulative 
degree days, and then
based on them produces some random numbers from Weibull
distribution and then generates (predicts/models) number of eggs,
larva, pupa and adults, concatenates the new data to the loaded data
and spits it out.
%%%%%%%%%%%%%%%

\item \func{CodlingMothPercentPopulation( CodMothParams, metdata\_data.table )}

\code{input:}
\begin{itemize}
\item \vari{CodMothParams} The parameters of the model.
\item \vari{metdata\_data.table} The collected data
\end{itemize}

\code{output:} This reads the data off the disk, and adds some
new columns to it, containing percentages of stuff, it seems. (I have to find the source
that this code is based on. Where did he learn the model from?)

\subsection{Script}
This goes from line 210 to 1400 (in the clean version).
\end{enumerate}

